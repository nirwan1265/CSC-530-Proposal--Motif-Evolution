
  \documentclass{lxaiproposal}
  
  
%   \usepackage[english,french]{babel}   % "babel.sty" + "french.sty"
   \usepackage[english]{babel}   % "babel.sty" + "french.sty"
   
% \usepackage[english,francais]{babel} % "babel.sty"
% \usepackage{french}                  % "french.sty"
  \usepackage{times}			% ajout times le 30 mai 2003
 
\usepackage{epsfig}
\usepackage{graphicx}
\usepackage{amsmath}
\usepackage{amssymb}
 \usepackage{booktabs}
 \usepackage{multirow}
\usepackage{pifont}
\usepackage{caption}
\usepackage{subcaption}



\usepackage{array}
\usepackage{color}
\usepackage{colortbl}

\usepackage{pifont}
\usepackage{amssymb}
\usepackage{latexsym}

\usepackage{booktabs}

%% --------------------------------------------------------------
%% FONTS CODING ?
% \usepackage[OT1]{fontenc} % Old fonts
% \usepackage[T1]{fontenc}  % New fonts (preferred)
%% ==============================================================

\title{Motif Evolution}

\author{\coord{Jaells}{Naranjo}{1} \\
       \coord{Nirwan}{Tandukar}{2}
      }

\address{\affil{1}{Department of Bioinformatics, NCSU}
         \affil{2}{Department of Functional Genomics, NCSU}}



\email{JN jgnaranj@ncsu.edi, NT ntanduk@ncsu.edu}




\begin{document}
\maketitle
%
\section{Project Description}
\vspace*{-3mm}

A little info about Transcriptional factor motif and genetic algorithm for motif finder. 
\textbf{(1-2 paragraphs)}

\section{Relevance and related works}
\vspace*{-3mm}

Some papers on genetic algorithm \cite{huo2010optimizing}. Some papers on HPC1 gene and transcription  factor \cite{barnes2022adaptive}. 

\textbf{(4-5 lines)}


\section{Methods and Approach}
\vspace*{-3mm}
Database to use:
Pseudo codes for random sequence generator and genetic algorithm



\textbf{(half a page)}


\section{Expected Results }
\vspace*{-3mm}

impact and outcomes your project will have. 
\textbf{(4-5 lines)}


\section{Timeline, Bottlenecks, and Contingency Plans}
\vspace*{-3mm}

Some other methods if things go wrong. 


\section{Data policy}
\vspace*{-3mm}

Our goal is to support work where the output will be made available to the broader research community. The source code will be freely provided and can be reached by anyone with our Github accounts. 

\textbf{(1 paragraph)} \\





 %\begin{figure} [h!]
 %    \centering
 %    \includegraphics[width=1\linewidth]{images/fig1.png}
 %  \caption{A typical figure.}
 %    \label{fig:my-fig}
 %    \end{figure}

\bibliographystyle{ieee_fullname}
\bibliography{references}

%\begin{thebibliography}{99}
%\end{thebibliography}

\end{document}
